\section{Authentication}
In order to complete the tasks that follow, you will need to provide Taikun credentials to Terraform.
As some of the tasks require Partner privileges, please use a Partner or Admin account.\\

We will not explain how to provide sensitive input variables to Terraform as those concerns are outside the scope of this
workshop. % or should we let people do this setup themselves?
To find out more about this topic, see this \href{https://learn.hashicorp.com/tutorials/terraform/sensitive-variables}{Hashicorp tutorial}.
A script is thus provided to manage Taikun authentication.
Run it and provide your credentials as follows. The script only needs to be run once.
\begin{shell}
./setup.sh
\end{shell}
\begin{raw}
Enter your Taikun email:
> jane.doe@itera.io
Enter your Taikun password:
> password123
\end{raw}


\section{Tasks}
\begin{warn}{Important}
  Throughout this workshop, in order to avoid conflicts and
  to permit easy clean up of leftover infrastructure, please follow the naming conventions used below.\\

  All resources names will follow the format \texttt{tfws-<n>-<firstname>-<s>}, where:
  \begin{itemize}
    \item \texttt{<n>} is the task number
    \item \texttt{<firstname>} is your firstname
    \item \texttt{<s>} is an arbitrary string of your choosing
  \end{itemize}
  For example, while completing task 9, Jane Doe might name their Taikun project
  \texttt{tfws-9-jane-mysuperproject}.
\end{warn}

\subsection{Task 0}
\begin{note}{Note}
For this task, please switch into the \texttt{task\_00} directory.
\end{note}
\blindtext{}
