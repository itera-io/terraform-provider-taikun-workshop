\subsection{Task 5}\label{sec:task5}

\begin{note}
For this task, please write your code in the file \texttt{task5.tf}
at the root of the \texttt{workshop/} directory.\\
You will also be editing \texttt{task5.auto.tfvars}.
\end{note}

Before we can create our first project, we need an access profile.
\href{https://intuinewin.github.io/taikun-docs/resources/access_profile.html}{Here is its documentation}.
This access profile will have multiple SSH users. Once again, we would like
to define those in a variable in order to avoid writing multiple \texttt{ssh\_user} blocks, as in the following example.
\begin{tf}
resource "taikun_access_profile" "foo" {
  # ...

  # This is what we want to avoid:
  ssh_user {
      name       = "alice"
      public_key = "ssh-ed25519 AAAATHEQUICKBROWNFOXJUMPSOVERTHELAZYDOG alice"
  }
  ssh_user {
      name       = "alice-home"
      public_key = "ssh-ed25519 AAAATHEQUICKBROWNFOXJUMPSOVERTHELAZYDOG alice-home"
  }
  ssh_user {
      name       = "bob"
      public_key = "ssh-ed25519 AAAATHEQUICKBROWNFOXJUMPSOVERTHELAZYDOG bob"
  }
  ssh_user {
      name       = "bob-home-laptop"
      public_key = "ssh-ed25519 AAAATHEQUICKBROWNFOXJUMPSOVERTHELAZYDOG bob-home-laptop"
  }
}
\end{tf}

You will use \texttt{for\_each} in a slightly different manner as \texttt{ssh\_user} is a nested resource
within the \texttt{access\_profile} resource.

