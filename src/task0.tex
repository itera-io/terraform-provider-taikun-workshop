\subsection{Task 0: Organization}\label{sec:task0}

\begin{note}
For this task, please write your code in the file \texttt{task0.tf}
at the root of the \texttt{workshop/} directory.
\end{note}

This objective of this first task is to create an organization.
All resources created in the future will be part of this organization.
As this is the first task, every step of the process is documented.\\

Before you do anything, start by preparing your working directory for other commands.
\begin{shell}
terraform init
\end{shell}
\begin{tip}
\texttt{terraform init} only needs to be run once when starting a new project or after updating
the list of providers to use.
\end{tip}
If all went well, you should see the following message.
\begin{raw}
Initializing the backend...
[...]
Terraform has been successfully initialized!
\end{raw}

Once Terraform has been initialized correctly, you can declare your organization resource.
Create \texttt{task0.tf} and write the following configuration block to it.
\begin{tf}
resource "taikun_organization" "myorg" {
  name          = "<name>"
  full_name     = "<full-name>"
  discount_rate = 120
}
\end{tf}
Be sure to replace \texttt{<name>} and \texttt{<full-name>} with
names of your choosing.
You can also choose another label instead of \texttt{myorg}.

\begin{tip}
Notice the syntax of the configuration block, as you are creating a resource,
it begins with the keyword \texttt{resource}, followed by its type between double quotes.
The type of resource is always lowercase and prefixed by the name of the provider,
thus \texttt{"taikun\_organization"}.
Following the resource's type is a label, it must be unique for this type of resource, and is used
to refer to this specific resource, as you will find out later.
Watch out, this label does not correspond to the name of the resource in Taikun.\\

Three arguments are then defined: \texttt{name}, \texttt{full\_name} and \texttt{discount\_rate}.
On the left side of the equals sign is the argument's identifier, on the right is its value.
See the
  \href{https://registry.terraform.io/providers/itera-io/taikun/latest/docs/resources/organization}{documentation
  of Taikun's organization resource} for a full list of arguments, i.e. the
  resource's \textit{schema}.\\

Labels and argument names can contain letters, digits, underscores and hyphens and may not start with a digit.
\end{tip}
Run the following command to reformat your configuration in the standard style.
\begin{shell}
terraform fmt
\end{shell}
Now apply your changes.
\begin{tip}
If you have already created resources, \texttt{apply} will refresh Terraform's state
by making request to Taikun's API. If you wish to check the validity of your changes
without refreshing the state, you can run \texttt{terraform validate}.
\end{tip}
\begin{shell}
terraform apply
\end{shell}
You should get a validation error.
\begin{raw}
Error: expected discount_rate to be in the range (0.000000 - 100.000000), got 120.000000
\end{raw}
Now fix the discount rate so it is in the range 0-100 and run \texttt{apply} once more.
Terraform will display a list of resources to create.
After checking the plan is correct, type \texttt{yes} to execute it.
\begin{tip}
You should notice a file \texttt{terraform.tfstate} in your working directory,
Terraform uses this file to keep track of the state, do not modify or delete it.
\end{tip}
You may now list the resources in Terraform's state.
\begin{shell}
terraform state list
\end{shell}
\begin{raw}
taikun_organization.myorg
\end{raw}
You can also display their content.
\begin{shell}
terraform show
\end{shell}
\begin{raw}
# taikun_organization.myorg:
resource "taikun_organization" "myorg" {
    created_at                       = "2021-11-05T14:00:50Z"
    discount_rate                    = 42
    full_name                        = "Jane Doe's organization"
    id                               = "6383"
    is_read_only                     = false
    lock                             = false
    managers_can_change_subscription = true
    name                             = "my-organization"
    partner_id                       = "119"
    partner_name                     = "TF-CI"
    projects                         = 0
    servers                          = 0
}
\end{raw}
You may wish to check the organization was indeed created at
\href{https://app.taikun.cloud/organizations}{app.taikun.cloud/organizations}.
\begin{tip}
Try running \texttt{terraform apply} again, Terraform will refresh its state, and, as long as
nothing has changed, tell you that no changes are needed.\\

You can also try deleting the organization through the web UI and running \texttt{terraform apply}.
Terraform will tell you that changes have occured outside of Terraform and recreate the resource.
\end{tip}
