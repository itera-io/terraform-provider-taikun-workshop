\subsection{Task 4}\label{sec:task4}

\begin{note}
For this task, please write your code in the file \texttt{task4.tf}
at the root of the \texttt{workshop/} directory.\\
You will also be editing \texttt{users.tf} and \texttt{users.auto.tfvars}.
\end{note}

We will now add users to the Taikun organization.
As we may want to define a large amount of users, we will use variables
and the keyword \texttt{for\_each} to avoid declaring multiple \texttt{taikun\_user}
blocks.\\

Add the following variable declaration to \texttt{users.tf}.
\begin{tf}
variable "users" {
  type = map(object({
    email = string
    role  = string
  }))
  description = "List of project users"
  default     = {}
}
\end{tf}

Here we are defining a complex type: a map of objects with two arguments, \texttt{email} and \texttt{role}.
The keys of the map are strings, they will be the usernames of the users.
The \texttt{default = \{\}} argument definition tells Terraform that the default value of \texttt{var.users} is an empty map.\\

Here is an example definition of the \texttt{users} variable.
\begin{tf}
users = {
  "alice" = {
    email = "alice@gmail.com"
    role  = "Manager"
  },
  "bob" = {
    email = "bob@gmail.com"
    role  = "User"
  },
}
\end{tf}
In this example, we are defining user accounts for Alice and Bob.
\begin{itemize}
  \item Alice has a Manager account with the username \texttt{alice} and the email \texttt{alice@gmail.com}.
  \item Bob has a User account with the username \texttt{bob} and the email \texttt{bob@gmail.com}.
\end{itemize}
Now edit \texttt{users.auto.tfvars} and define three users.
\begin{itemize}
  \item \texttt{tfws-<firstname>-manager} with Manager role and the email \texttt{tfws-<firstname>-manager@mail.example}.
  \item \texttt{tfws-<firstname>-user1} with User role and the email \texttt{tfws-<firstname>-user1@mail.example}.
  \item \texttt{tfws-<firstname>-user2} with User role and the email \texttt{tfws-<firstname>-user2@mail.example}.
\end{itemize}
As always, replace \texttt{<firstname>} with your firstname.\\

You can now declare the user resource in \texttt{task4.tf}, your users must belong to the organization you created in \fullref{sec:task0}.
\href{https://intuinewin.github.io/taikun-docs/resources/user.html}{Here} is its documentation.
By using the \texttt{for\_each} keyword, you will only need to define one resource block.

\begin{tip}
  To understand the \texttt{for\_each} keyword, let us consider a Terraform provider to order pizzas.
  Suppose we define the following map variable, \texttt{pizza\_orders}.
  \begin{tf}
pizza_orders = {
  "alice" = {
    type = "pepperoni"
    size = "large"
  },
  "bob" = {
    type = "amatriciana"
    size = "medium"
  },
}
  \end{tf}
  Here is how we would use it and the \texttt{for\_each} keyword in a \texttt{pizza\_order} resource.
  \begin{tf}
resource "pizza_order" "orders" {
  for_each = var.pizza_orders

  client = each.key
  type   = each.value.type
  size   = each.value.size
}
  \end{tf}

  You can also have a look at Terraform's \href{https://www.terraform.io/docs/language/meta-arguments/for_each.html}{for\_each documentation}.
\end{tip}

Once you have declared the user resource, apply your changes and move on to the next task.
