\section{Setup}

You can choose to install Terraform and the Taikun provider on your machine and
then download the workshop files.
However, if you wish to use the prebuilt Docker image which already contains
the workshop files, skip ahead to \fullref{sec:docker}.

\subsection{Terraform installation}
You must have Terraform version 0.14 or newer installed.
\subsubsection{MacOS with Homebrew}
\begin{shell}
brew tap hashicorp/tap
brew update
brew install hashicorp/tap/terraform
\end{shell}

\subsubsection{Ubuntu/Debian with Apt}
\begin{shell}
sudo apt-get update && sudo apt-get install -y gnupg software-properties-common curl
curl -fsSL https://apt.releases.hashicorp.com/gpg | sudo apt-key add -
sudo apt-add-repository \
"deb [arch=amd64] https://apt.releases.hashicorp.com `lsb_release -cs` main"
sudo apt-get update && sudo apt-get install terraform
\end{shell}

\subsubsection{Archlinux with Pacman}
\begin{shell}
sudo pacman -S terraform
\end{shell}

\subsubsection{Manually}
Download the proper package from the \href{https://www.terraform.io/downloads.html}{Download Terraform} page and unzip it to extract the \texttt{terraform} binary.\\
List the directories in your \texttt{PATH}.
\begin{shell}
echo $PATH
\end{shell}
Place the \texttt{terraform} binary in one of the listed directories (avoid putting it in \texttt{/usr/bin} or \texttt{/usr/local/sbin}).
For example:
\begin{raw}
$ echo $PATH
/usr/local/sbin:/usr/local/bin:/usr/bin:/usr/bin/site_perl
$ sudo mv ~/Downloads/terraform /usr/local/bin/
\end{raw}

\subsection{Installing the Taikun provider}
Until the provider is listed on Terraform's \href{https://registry.terraform.io/browse/providers}{plugin registry}, the provider must be installed locally.
Start by cloning the GitHub repo.
You will need to join the \href{https://github.com/itera-io/}{itera-io GitHub organization}
as the repo is currently private.
\begin{shell}
git clone git@github.com:itera-io/terraform-provider-taikun.git
\end{shell}
Then run the following commands to install the provider on your machine.
\begin{shell}
cd terraform-provider-taikun
make dockerinstall
\end{shell}
The provider will be installed at:\\
\texttt{~/.terraform.d/plugins/itera-io/dev/taikun/0.1.0/linux\_amd64/terraform-provider-taikun}.

\subsection{Downloading the workshop files}
Terraform config files with the minimal required configuration are provided for each task.
Download the \texttt{workshop.zip} file from [insert method of distribution] % slack channel? git repo?
and extract its contents in the directory of your choice.\\

To do so from the command line, type the following command.
\begin{shell}
unzip workshop.zip
\end{shell}
This will extract the \texttt{workshop} directory into your working directory.
