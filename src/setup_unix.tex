\section{Local setup}

If you are on Linux or MacOS,
you can choose to install Terraform and the Taikun provider on your machine and
then download the workshop files.
However, if you wish to use the prebuilt Docker image which already contains
the workshop files skip ahead to \fullref{sec:docker}.

\begin{warn}
If your are using Windows, skip ahead to \fullref{sec:docker}.
\end{warn}

\subsection{Requirements}
\begin{itemize}
  \item You must have Terraform version 0.14 or newer installed.
  \item You will need Git to clone the provider's repo.
  \item To build the provider, you will need either Docker or
    the latest stable version of \href{https://golang.org/dl/}{Go}.
\end{itemize}

\subsection{Installing Terraform}
You can install Terraform using one of the popular package managers, or manually by
download the binary.
\subsubsection{Homebrew on MacOS}
\begin{shell}
brew tap hashicorp/tap
brew update
brew install hashicorp/tap/terraform
\end{shell}

\subsubsection{Apt on Ubuntu or Debian}
\begin{shell}
sudo apt-get update && sudo apt-get install -y gnupg software-properties-common curl
curl -fsSL https://apt.releases.hashicorp.com/gpg | sudo apt-key add -
sudo apt-add-repository \
"deb [arch=amd64] https://apt.releases.hashicorp.com `lsb_release -cs` main"
sudo apt-get update && sudo apt-get install terraform
\end{shell}

\subsubsection{Pacman on Archlinux}
\begin{shell}
sudo pacman -S terraform
\end{shell}

\subsubsection{Manually}
Download the proper package from the \href{https://www.terraform.io/downloads.html}{Download Terraform page}
and unzip it to extract the \texttt{terraform} binary.
Then place the binary in your \texttt{PATH}.\\

Run the following command to display a colon-separated list of directories in your \texttt{PATH}.
\begin{shell}
echo $PATH
\end{shell}
This might output the following.
\begin{raw}
/usr/local/sbin:/usr/local/bin:/usr/bin:/usr/bin/site_perl
\end{raw}
Place the \texttt{terraform} binary in one of the listed directories (avoid putting it in \texttt{/usr/bin} or \texttt{/usr/local/sbin} if you can).
For example:
\begin{shell}
sudo mv ~/Downloads/terraform /usr/local/bin/
\end{shell}

\subsection{Installing the Taikun provider}
Until the provider is listed on Terraform's \href{https://registry.terraform.io/browse/providers}{plugin registry},
the provider must be installed locally.\\

Start by cloning its GitHub repo and switching into it.
\begin{shell}
git clone https://github.com/itera-io/terraform-provider-taikun.git
cd terraform-provider-taikun
\end{shell}
If you have the Go toolchain installed,
run the following command to build and install the provider on your machine.
\begin{shell}
make install
\end{shell}
Or if you wish to build it using docker, run:
\begin{shell}
make dockerinstall
\end{shell}
The provider will be installed at the following location.\\
\texttt{~/.terraform.d/plugins/itera-io/dev/taikun/0.1.0/linux\_amd64/terraform-provider-taikun}.

\subsection{Downloading the workshop files}
Clone the workshop directory and switch into the \texttt{workshop/} directory.
\begin{shell}
git clone https://github.com/itera-io/terraform-provider-taikun-workshop.git
cd terraform-provider-taikun-workshop/workshop/
\end{shell}
