\subsection{Task 3}\label{sec:task3}

\begin{note}
For this task, please write your code in the file \texttt{task3.tf}
at the root of the \texttt{workshop/} directory.\\
You will also be editing \texttt{task3.auto.tfvars}.
\end{note}

In order to create a Taikun project, we will need cloud credentials.
You will be using the OpenStack credentials provided on the \textbf{\#terraform-workshop}
Slack channel. In a real work environment, we wouldn't want to store our cloud credentials in version control,
this is where input variables come into play.\\

Variables are declared in \texttt{variables.tf} by convention.
If you have a look at its contents,
this file declares the variables \texttt{taikun\_email} and
\texttt{taikun\_password} used for authentication.
\begin{tf}
# variables.tf
variable "taikun_email" {
  description = "Taikun email"
  type        = string
  sensitive   = true
}

variable "taikun_password" {
  description = "Taikun password"
  type        = string
  sensitive   = true
}
\end{tf}

\begin{tip}
  Input variables are declared with a \texttt{variable} block.
  The label that follows the \texttt{variable} keyword is the name of the variable.

  \begin{itemize}
    \item The \texttt{description} argument is used to specifiy the variable's documentation.
    \item \texttt{type} is the type of this argument's value.
    \item If set to \texttt{true}, \texttt{sensitive} will hide this variable's value in Terraform output. It defaults to false.
  \end{itemize}

  To know more about input variables and a full list of arguments,
  see the \href{https://www.terraform.io/docs/language/values/variables.html}{Terraform documentation on variables}.\\

  Variables are then defined in \texttt{.tfvars} files, as you saw in \fullref{sec:auth}.\\
\end{tip}

For the sake of simplicity, variables will be declared in the \texttt{taskN.tf} files.
Thus, in \texttt{task3.tf}, insert the following variable declarations.
\begin{tf}
variable "openstack_url" {
  description = "OpenStack url"
  type        = string
  sensitive   = true
}
variable "openstack_user" {
  description = "OpenStack user"
  type        = string
  sensitive   = true
}
variable "openstack_password" {
  description = "OpenStack password"
  type        = string
  sensitive   = true
}
variable "openstack_domain" {
  description = "OpenStack domain"
  type        = string
  sensitive   = true
}
variable "openstack_region" {
  description = "OpenStack region"
  type        = string
}
variable "openstack_public_network" {
  description = "OpenStack public network"
  type        = string
}
variable "openstack_project" {
  description = "OpenStack project name"
  type        = string
}
\end{tf}
\begin{note}
  If copy-pasting this code block into \texttt{task3.tf} does not indent the code properly,
  simply save \texttt{task3.tf} and run \texttt{terraform fmt}.
\end{note}

Now that the OpenStack variables have been declared,
define the variables using the credentials provided on \textbf{\#terraform-workshop}, in \texttt{task3.auto.tfvars}.

\begin{tip}
As a reminder, here is the syntax used in \texttt{taikun\_auth.auto.tfvars} to define the variables \texttt{taikun\_email}
and \texttt{taikun\_password}.
\begin{tf}
taikun_email = "jane.doe@itera.io"
taikun_password = "PassWord123"
\end{tf}
\end{tip}

\begin{tip}
  Terraform must be told through command line arguments which \texttt{.tfvars} files to read.
  However, if variable definition files have the extension \texttt{.auto.tfvars}, as is the case with
  \texttt{taikun\_auth.auto.tfvars}, Terraform will automatically fetch the variables' values.
\end{tip}

Now that your OpenStack variables have been declared and defined, you can define the OpenStack cloud credential resource in \texttt{task3.tf}.
\href{https://intuinewin.github.io/taikun-docs/resources/cloud_credential_openstack.html}{Here} is its documentation.

\begin{tip}
In order to get a variable's value, use the syntax \texttt{var.variable\_name}.
For example, to set the OpenStack domain in your \texttt{cloud\_credential\_openstack} resource,
  you will write the following. \mint{terraform}|domain = var.openstack_domain|
\end{tip}

Once you have declared your new resource, apply your changes and move on to the next task.

\begin{warn}
  \begin{itemize}
    \item Remember to follow the \texttt{tfws-<firstname>-<s>} naming convention when naming your cloud credential resource.
    \item As always, your resources should belong to the organization you created in \fullref{sec:task0}.
  \end{itemize}
\end{warn}

