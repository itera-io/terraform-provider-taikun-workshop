\section{Setup using Docker}\label{sec:docker}

You can choose to build and use a Docker image to complete this workshop.

\subsection{Requirements}

You must have Docker and Git installed.\\
You need to clone the workshop repository which contains all the files you'll need.\\
In order to clone this repository, you'll need a Github account and to join the \href{https://github.com/itera-io}{Itera organization} on Github.

\begin{shell}
git clone --recursive git@github.com:itera-io/terraform-provider-taikun-workshop.git
\end{shell}

\subsection{Docker image creation}

First, you'll need to build the image, this operation can take several minutes.

\begin{shell}
docker build -t tf-workshop .
\end{shell}

\subsection{Docker container creation}
To create the Docker container, run the following command.
\begin{shell}
docker run --name tf-workshop -it tf-workshop
\end{shell}
You are now logged in to the container as root. The workshop files,
Terraform and the Taikun Terraform provider are already installed.\\

You can exit the container without losing your progress; simply restart it with:
\begin{shell}
docker start -i tf-workshop
\end{shell}
As you will need to edit text files, the Docker image has the \texttt{vim},
\texttt{emacs}, \texttt{micro} and \texttt{nano} packages installed.
If you are unfamiliar with Emacs and Vim keybindings, the Micro editor
has traditional Common User Access keybindings (\texttt{Ctrl-C} for copy,
\texttt{Ctrl-Z} for undo, etc).
