\subsection{Setup using Docker}\label{sec:docker}

\subsubsection{Requirements}
You must have Docker and Git installed.

\subsubsection{Setup}
Start by cloning the workshop repository.
\begin{shell}
git clone --recursive https://github.com/itera-io/terraform-provider-taikun-workshop.git
\end{shell}

\subsubsection{Docker image creation}
You'll need to build the image, this operation can take several minutes.
\begin{shell}
DOCKER_BUILDKIT=1 docker build --rm --target bin -t tf-workshop .
\end{shell}

\subsubsection{Docker container creation}
To create the Docker container, run one of the following commands from the root of \\
\texttt{terrraform-provider-taikun-workshop}.
\begin{itemize}
  \item On Windows, run the following command in the command prompt (not Powershell).
  \begin{cmd}
docker run -v %CD%\workshop:/root/workshop --name tf-workshop -it tf-workshop
  \end{cmd}
  \item On Linux and MacOS:
  \begin{shell}
docker run -v $(pwd)/workshop:/root/workshop --name tf-workshop -it tf-workshop
  \end{shell}
\end{itemize}
This will mount the \texttt{workshop/} directory in the Docker container and
log you in to the container as \texttt{root}.
In other words, you will need to run \texttt{terraform} commands from within
the container shell, however you can edit the files in \texttt{workshop/}
with the editor of your choice on your machine.\\

\subsubsection{Restarting the Docker container}
You can exit the container at any time without losing your progress.
If you have exited the Docker container, run the following command to restart it.
\begin{shell}
docker start -i tf-workshop
\end{shell}

\subsubsection{Text editing within the container}
If you wish to edit the files in \texttt{workshop/} from within the container,
the Docker image has the \texttt{vim}, \texttt{micro} and \texttt{nano} packages installed.
If you are unfamiliar with Nano and Vim keybindings, the Micro editor
has traditional Common User Access keybindings (\texttt{Ctrl-C} for copy,
\texttt{Ctrl-Z} for undo, etc).
