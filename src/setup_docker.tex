\section{Setup using Docker}\label{sec:docker}

\subsection{Requirements}
\begin{itemize}
  \item You must have Docker and Git installed.
  \item To clone the required repositories, you need to join Itera's \href{https://github.com/itera-io}{GitHub organization}.
\end{itemize}

\subsection{Setup}
Start by cloning the workshop repository.
\begin{shell}
git clone --recursive git@github.com:itera-io/terraform-provider-taikun-workshop.git
\end{shell}

\subsection{Docker image creation}
You'll need to build the image, this operation can take several minutes.
\begin{shell}
docker build -t tf-workshop .
\end{shell}

\subsection{Docker container creation}
To create the Docker container, run the following command.
\begin{shell}
docker run --name tf-workshop -it tf-workshop
\end{shell}
You are now logged in to the container as root. The workshop files,
Terraform and the Taikun Terraform provider are already installed.
You can exit the container at any time without losing your progress.

\subsection{Restarting the Docker container}
If you have exited the Docker container, run the following command to restart it.
\begin{shell}
docker start -i tf-workshop
\end{shell}

\subsection{Text editing}
As you will need to edit text files on the command line,
the Docker image has the \texttt{vim},
\texttt{emacs}, \texttt{micro} and \texttt{nano} packages installed.
If you are unfamiliar with Emacs and Vim keybindings, the Micro editor
has traditional Common User Access keybindings (\texttt{Ctrl-C} for copy,
\texttt{Ctrl-Z} for undo, etc).\\

If you don't want to use the command line, you can use Visual Studio Code with
the Remote Development extension to edit files and run commands inside the
container.