\section{Documentation}
Once the provider has been added to the Terraform Registry, the documentation will be available online there.
For the purpose of this workshop, the documentation is being temporarily hosted on \href{https://intuinewin.github.io/taikun-docs/}{Github Pages}.

\section{Tasks}
The ultimate goal of this workshop is to have an operational Taikun project built solely with Terraform
configuration files.
By following a step by step process, you will discover how various Taikun
resources are declared and managed using Terraform.\\

All your work will be done in the \texttt{workshop/} directory.
\begin{raw}
./workshop/
|-- main.tf
|-- taikun_auth.auto.tfvars
|-- variables.tf
\end{raw}
\texttt{main.tf} contains the Provider configuration, namely where it is installed
and what credentials to use. You will not need to edit this file.
\begin{tf}
# main.tf
terraform {
  required_providers {
    taikun = {
      source = "itera-io/dev/taikun"
    }
  }
}

provider "taikun" {
  email    = var.taikun_email
  password = var.taikun_password
}
\end{tf}
In fact, Terraform reads its configuration from all the files with the extension \texttt{.tf},
in the working directory. The extension \texttt{.tfvars} will be discussed later.
Having the provider configuration in \texttt{main.tf} is simply a convention.
During this workshop, each task should be coded in a separate config file,
such that, at the end of the workshop, your directory is organized as such:
\begin{raw}
./workshop/
|-- main.tf
|-- taikun_auth.auto.tfvars
|-- task0.tf
|-- task1.tf
|-- task2.tf
|-- task3.auto.tfvars
|-- task3.tf
|-- task4.tf
|-- task5.tf
|-- task6.tf
|-- variables.tf
|-- users.auto.tfvars
|-- users.tf
\end{raw}
% TODO update as tasks are added

\begin{warn}
  Throughout this workshop, in order to avoid conflicts and
  to permit easy clean up of leftover infrastructure, please follow the naming conventions used below.\\

  All resources names will follow the format \texttt{tfws-<firstname>-<s>}, where:
  \begin{itemize}
    \item \texttt{<firstname>} is your firstname
    \item \texttt{<s>} is an arbitrary string of your choosing
  \end{itemize}
  For example, Jane Doe might name their Taikun project \texttt{tfws-jane-mysuperproject}.
\end{warn}

\subsection{Authentication}\label{sec:auth}
In order to complete the tasks that follow, you will need to provide Taikun credentials to Terraform.
You will be using a Partner account, as some of the tasks, such as creating an organization,
require Partner privileges.\\

We will explain how input variables work later in the workshop.
For now, simply edit \texttt{taikun\_auth.auto.tfvars}
and replace the values of \texttt{taikun\_email} and \texttt{taikun\_password}
with the credentials provided on the \textbf{\#terraform-workshop} Slack channel.
\begin{tf}
# taikun_auth.auto.tfvars
taikun_email = "jane.doe@itera.io"
taikun_password = "PassWord123"
\end{tf}
To find out more about providing sensitive data to Terraform, see this \href{https://learn.hashicorp.com/tutorials/terraform/sensitive-variables}{Hashicorp tutorial}.

\subsection{Task 0: Organization}\label{sec:task0}

\begin{note}
For this task, please write your code in the file \texttt{task0.tf}
at the root of the \texttt{workshop/} directory.
\end{note}

This objective of this first task is to create an organization.
All resources created in the future will be part of this organization.
As this is the first task, every step of the process is documented.\\

Before you do anything, start by preparing your working directory for other commands.
\begin{shell}
terraform init
\end{shell}
\begin{tip}
\texttt{terraform init} only needs to be run once when starting a new project or after updating
the list of providers to use.
\end{tip}
If all went well, you should see the following message.
\begin{raw}
Initializing the backend...
[...]
Terraform has been successfully initialized!
\end{raw}

Once Terraform has been initialized correctly, you can declare your organization resource.
Create \texttt{task0.tf} and write the following configuration block to it.
\begin{tf}
resource "taikun_organization" "myorg" {
  name          = "<name>"
  full_name     = "<full-name>"
  discount_rate = 120
}
\end{tf}
Be sure to replace \texttt{<name>} and \texttt{<full-name>} with
names of your choosing.
You can also choose another label instead of \texttt{myorg}.

\begin{tip}
Notice the syntax of the configuration block, as you are creating a resource,
it begins with the keyword \texttt{resource}, followed by its type between double quotes.
The type of resource is always lowercase and prefixed by the name of the provider,
thus \texttt{"taikun\_organization"}.
Following the resource's type is a label, it must be unique for this type of resource, and is used
to refer to this specific resource, as you will find out later.
Watch out, this label does not correspond to the name of the resource in Taikun.\\

Three arguments are then defined: \texttt{name}, \texttt{full\_name} and \texttt{discount\_rate}.
On the left side of the equals sign is the argument's identifier, on the right is its value.
See the
  \href{https://registry.terraform.io/providers/itera-io/taikun/latest/docs/resources/organization}{documentation
  of Taikun's organization resource} for a full list of arguments, i.e. the
  resource's \textit{schema}.\\

Labels and argument names can contain letters, digits, underscores and hyphens and may not start with a digit.
\end{tip}
Run the following command to reformat your configuration in the standard style.
\begin{shell}
terraform fmt
\end{shell}
Now apply your changes.
\begin{tip}
If you have already created resources, \texttt{apply} will refresh Terraform's state
by making request to Taikun's API. If you wish to check the validity of your changes
without refreshing the state, you can run \texttt{terraform validate}.
\end{tip}
\begin{shell}
terraform apply
\end{shell}
You should get a validation error.
\begin{raw}
Error: expected discount_rate to be in the range (0.000000 - 100.000000), got 120.000000
\end{raw}
Now fix the discount rate so it is in the range 0-100 and run \texttt{apply} once more.
Terraform will display a list of resources to create.
After checking the plan is correct, type \texttt{yes} to execute it.
\begin{tip}
You should notice a file \texttt{terraform.tfstate} in your working directory,
Terraform uses this file to keep track of the state, do not modify or delete it.
\end{tip}
You may now list the resources in Terraform's state.
\begin{shell}
terraform state list
\end{shell}
\begin{raw}
taikun_organization.myorg
\end{raw}
You can also display their content.
\begin{shell}
terraform show
\end{shell}
\begin{raw}
# taikun_organization.myorg:
resource "taikun_organization" "myorg" {
    created_at                       = "2021-11-05T14:00:50Z"
    discount_rate                    = 42
    full_name                        = "Jane Doe's organization"
    id                               = "6383"
    is_read_only                     = false
    lock                             = false
    managers_can_change_subscription = true
    name                             = "my-organization"
    partner_id                       = "119"
    partner_name                     = "TF-CI"
    projects                         = 0
    servers                          = 0
}
\end{raw}
You may wish to check the organization was indeed created at
\href{https://app.taikun.cloud/organizations}{app.taikun.cloud/organizations}.
\begin{tip}
Try running \texttt{terraform apply} again, Terraform will refresh its state, and, as long as
nothing has changed, tell you that no changes are needed.\\

You can also try deleting the organization through the web UI and running \texttt{terraform apply}.
Terraform will tell you that changes have occured outside of Terraform and recreate the resource.
\end{tip}

\subsection{Task 1: Kubernetes Profile}\label{sec:task1}

\begin{note}
For this task, please write your code in the file \texttt{task1.tf}
at the root of the \texttt{workshop/} directory.
\end{note}

Now that you have created an organization,
you will create a Kubernetes profile belonging to it.
Check the \texttt{kubernetes\_profile} resource's schema on
\href{https://registry.terraform.io/providers/itera-io/taikun/latest/docs/resources/kubernetes_profile}{the provider's documentation}
and declare the resource in \texttt{task1.tf}.
Set \texttt{organization\_id} to the ID
of the organization created in the previous task (see \fullref{sec:task0}).\\

Feel free to set some of \texttt{kubernetes\_profile}'s other optional attributes,
such as \texttt{schedule\_on\_master} and \texttt{load\_balancing\_solution}.

Once you have declared your resource, apply your changes and move on to the next task.

\begin{tip}
To refer to the ID of your organization, use the following syntax.
\begin{tf}
resource.taikun_organization.myorg.id
\end{tf}
Make sure to replace \texttt{myorg} if you used another label for your organization.
\end{tip}


\subsection{Task 2: Slack Configuration \& Alerting Profile}\label{sec:task2}

\begin{note}
For this task, please write your code in the file \texttt{task2.tf}
at the root of the \texttt{workshop/} directory.
\end{note}

You will now create an alerting profile using a Slack configuration.

\begin{enumerate}
  \item Start by declaring a Slack configuration.
  \href{https://registry.terraform.io/providers/itera-io/taikun/latest/docs/resources/slack_configuration}{Here} is its documentation.\\

Its hook URL should be \texttt{https://hooks.myapp.example/ci}.
It must send \textbf{alert-type notifications only} to the channel \texttt{ci}.
  \item You can now declare the alerting profile.
    \href{https://registry.terraform.io/providers/itera-io/taikun/latest/docs/resources/alerting_profile}{Here} is its documentation.\\
    The alerting profile should send notifications \textbf{daily} using the Slack configuration declared above.
\end{enumerate}

\begin{warn}
As always, your resources should belong to the organization created in
\fullref{sec:task0}.
\end{warn}

Once you have declared these two new resources, apply your changes and move on to the next task.

\subsection{Task 3}\label{sec:task3}

\begin{note}
For this task, please write your code in the file \texttt{task3.tf}
at the root of the \texttt{workshop/} directory.\\
You will also be editing \texttt{task3.auto.tfvars}.
\end{note}

In order to create a Taikun project, we will need cloud credentials.
You will be using the OpenStack credentials provided on the \textbf{\#terraform-workshop}
Slack channel. In a real work environment, we wouldn't want to store our cloud credentials in version control,
this is where input variables come into play.\\

Variables are declared in \texttt{variables.tf} by convention.
If you have a look at its contents,
this file declares the variables \texttt{taikun\_email} and
\texttt{taikun\_password} used for authentication.
\begin{tf}
# variables.tf
variable "taikun_email" {
  description = "Taikun email"
  type        = string
  sensitive   = true
}

variable "taikun_password" {
  description = "Taikun password"
  type        = string
  sensitive   = true
}
\end{tf}

\begin{tip}
  Input variables are declared with a \texttt{variable} block.
  The label that follows the \texttt{variable} keyword is the name of the variable.

  \begin{itemize}
    \item The \texttt{description} argument is used to specifiy the variable's documentation.
    \item \texttt{type} is the type of this argument's value.
    \item If set to \texttt{true}, \texttt{sensitive} will hide this variable's value in Terraform output. It defaults to false.
  \end{itemize}

  To know more about input variables and a full list of arguments,
  see the \href{https://www.terraform.io/docs/language/values/variables.html}{Terraform documentation on variables}.\\

  Variables are then defined in \texttt{.tfvars} files, as you saw in \fullref{sec:auth}.\\
\end{tip}

For the sake of simplicity, variables will be declared in the \texttt{taskN.tf} files.
Thus, in \texttt{task3.tf}, insert the following variable declarations.
\begin{tf}
variable "openstack_url" {
  description = "OpenStack url"
  type        = string
  sensitive   = true
}
variable "openstack_user" {
  description = "OpenStack user"
  type        = string
  sensitive   = true
}
variable "openstack_password" {
  description = "OpenStack password"
  type        = string
  sensitive   = true
}
variable "openstack_domain" {
  description = "OpenStack domain"
  type        = string
  sensitive   = true
}
variable "openstack_region" {
  description = "OpenStack region"
  type        = string
}
variable "openstack_public_network" {
  description = "OpenStack public network"
  type        = string
}
variable "openstack_project" {
  description = "OpenStack project name"
  type        = string
}
\end{tf}
\begin{note}
  If copy-pasting this code block into \texttt{task3.tf} does not indent the code properly,
  simply save \texttt{task3.tf} and run \texttt{terraform fmt}.
\end{note}

Now that the OpenStack variables have been declared,
define the variables using the credentials provided on \textbf{\#terraform-workshop}, in \texttt{task3.auto.tfvars}.\\

Terraform must be told through command line arguments which \texttt{.tfvars} files to read.
However, if variable definition files have the extension \texttt{.auto.tfvars}, as is the case with
\texttt{taikun\_auth.auto.tfvars}, Terraform will automatically fetch the variables' values.
\begin{tip}
As a reminder, here is the syntax used in \texttt{taikun\_auth.auto.tfvars} to define the variables \texttt{taikun\_email}
and \texttt{taikun\_password}.
\begin{tf}
taikun_email = "jane.doe@itera.io"
taikun_password = "PassWord123"
\end{tf}
\end{tip}

Now that your OpenStack variables have been declared and defined, you can define the OpenStack cloud credential resource in \texttt{task3.tf}.
\href{https://intuinewin.github.io/taikun-docs/resources/cloud_credential_openstack.html}{Here} is its documentation.

\begin{tip}
In order to get a variable's value, use the syntax \texttt{var.variable\_name}.
For example, to set the OpenStack domain in your \texttt{cloud\_credential\_openstack} resource,
  you will write the following. \mint{terraform}|domain = var.openstack_domain|
\end{tip}

Once you have declared your new resource, apply your changes and move on to the next task.

\begin{warn}
  \begin{itemize}
    \item Remember to follow the \texttt{tfws-<firstname>-<s>} naming convention when naming your cloud credential resource.
    \item As always, your resources should belong to the organization you created in \fullref{sec:task0}.
  \end{itemize}
\end{warn}


\subsection{Task 4}\label{sec:task4}

\begin{note}
For this task, please write your code in the file \texttt{task4.tf}
at the root of the \texttt{workshop/} directory.\\
You will also be editing \texttt{users.tf} and \texttt{users.auto.tfvars}.
\end{note}

We will now add users to the Taikun organization.
As we may want to define a large amount of users, we will use variables
and the keyword \texttt{for\_each} to avoid declaring multiple \texttt{taikun\_user}
blocks.\\

Add the following variable declaration to \texttt{users.tf}.
\begin{tf}
variable "users" {
  type = map(object({
    email = string
    role  = string
  }))
  description = "List of project users"
  default     = {}
}
\end{tf}

Here we are defining a complex type: a map of objects with two arguments, \texttt{email} and \texttt{role}.
The keys of the map are strings, they will be the usernames of the users.
The \texttt{default = \{\}} argument definition tells Terraform that the default value of \texttt{var.users} is an empty map.\\

Here is an example definition of the \texttt{users} variable.
\begin{tf}
users = {
  "alice" = {
    email = "alice@gmail.com"
    role  = "Manager"
  },
  "bob" = {
    email = "bob@gmail.com"
    role  = "User"
  },
}
\end{tf}
In this example, we are defining user accounts for Alice and Bob.
\begin{itemize}
  \item Alice has a Manager account with the username \texttt{alice} and the email \texttt{alice@gmail.com}.
  \item Bob has a User account with the username \texttt{bob} and the email \texttt{bob@gmail.com}.
\end{itemize}
Now edit \texttt{users.auto.tfvars} and define three users.
\begin{itemize}
  \item \texttt{tfws-<firstname>-manager} with Manager role and the email \texttt{tfws-<firstname>-manager@mail.example}.
  \item \texttt{tfws-<firstname>-user1} with User role and the email \texttt{tfws-<firstname>-user1@mail.example}.
  \item \texttt{tfws-<firstname>-user2} with User role and the email \texttt{tfws-<firstname>-user2@mail.example}.
\end{itemize}
As always, replace \texttt{<firstname>} with your firstname.\\

You can now declare the user resource in \texttt{task4.tf}, your users must belong to the organization you created in \fullref{sec:task0}.
\href{https://intuinewin.github.io/taikun-docs/resources/user.html}{Here} is its documentation.
By using the \texttt{for\_each} keyword, you will only need to define one resource block.

\begin{tip}
  To understand the \texttt{for\_each} keyword, let us consider a Terraform provider to order pizzas.
  Suppose we define the following map variable, \texttt{pizza\_orders}.
  \begin{tf}
pizza_orders = {
  "alice" = {
    type = "pepperoni"
    size = "large"
  },
  "bob" = {
    type = "amatriciana"
    size = "medium"
  },
}
  \end{tf}
  Here is how we would use it and the \texttt{for\_each} keyword in a \texttt{pizza\_order} resource.
  \begin{tf}
resource "pizza_order" "orders" {
  for_each = var.pizza_orders

  client = each.key
  type   = each.value.type
  size   = each.value.size
}
  \end{tf}

  You can also have a look at Terraform's \href{https://www.terraform.io/docs/language/meta-arguments/for_each.html}{for\_each documentation}.
\end{tip}

Once you have declared the user resource, apply your changes and move on to the next task.

\subsection{Task 5}\label{sec:task5}

\begin{note}
For this task, please write your code in the file \texttt{task5.tf}
at the root of the \texttt{workshop/} directory.\\
You will also be editing \texttt{task5.auto.tfvars}.
\end{note}


\subsection{Task 6: Project}\label{sec:task6}

\begin{note}
For this task, please write all your code in the file \texttt{task6.tf}
at the root of the \texttt{workshop/} directory.
\end{note}

Finally, you can declare a project resource.
However, as flavors must be bound to the project,
you must first fetch a list of suitable flavors.\\

To do this, declare a \href{https://intuinewin.github.io/taikun-docs/data-sources/flavors.html}{flavors datasource}.
Datasources, as opposed to resources, only fetch information from providers and do not create any resources.
Add the following block to \texttt{task6.tf}.
\begin{tf}
data "taikun_flavors" "small" {
  # FIXME
}
\end{tf}

\begin{tip}
As you are declaring a datasource and not a resource, the block begins with the keyword \texttt{data} instead of \texttt{resource}.
Once again, the type of datasource is in lowercase and must be prefixed by the name of the provider.
Finally, the label \texttt{"small"} is used to designate this datasource.
\end{tip}

Edit the datasource to search for flavors with 4 or fewer CPUs and no more than 8GB of RAM.
Set its cloud credential ID to that of the cloud credential created in \fullref{sec:task3}.\\

Then declare a local value \texttt{flavors} to be the list of names of the flavors read
by the datasource.
See the \href{https://www.terraform.io/docs/language/values/locals.html}{Terraform documentation}
to know more about local values.
\begin{tf}
locals {
  flavors = [for flavor in data.taikun_flavors.small.flavors : flavor.name]
}
\end{tf}
This will allow you to refer to the list of flavor names with \texttt{local.flavors},
which will be useful when defining the project.\\

You now have everything you need to create a project in Taikun.
\href{https://intuinewin.github.io/taikun-docs/resources/project.html}{Here} is its documentation.\\

These are the requirements for the project resource:
\begin{itemize}
  \item As all previous resources, it must belong to the organization created in \fullref{sec:task0}.
  \item It must use the kubernetes profile defined in \fullref{sec:task1}.
  \item Its alerting profile must be the one defined \fullref{sec:task2}.
  \item It should use the cloud credentials defined in \fullref{sec:task3}.
  \item Monitoring must be enabled.
  \item It should have one bastion, one kubemaster and one kubeworker.
\end{itemize}

\begin{tip}
  To access the first element of a list, use the syntax \texttt{list[index]}.
  For example, to get the first flavor read by the \texttt{flavors} datasource,
  use \texttt{local.flavors[0]}.
\end{tip}
\pagebreak

Once you have declared the project resource, go ahead and apply your changes.
\begin{hint}
  The Taikun provider was built with the knowledge that you are probably getting hungry at this point.
  Thus, you can expect to wait about 30 minutes for your project to be in \texttt{Ready} state.
  Please take this time to enjoy the refreshments available in the office, or whatever is
  in your fridge if you are working from home.
\end{hint}

