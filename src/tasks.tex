\section{Documentation}
The provider documentation is available on the \href{https://registry.terraform.io/providers/itera-io/taikun/latest/docs}{Terraform Registry}.

\section{Tasks}
The end goal of this workshop is to have an operational Taikun project built solely with Terraform
configuration files.
By following a step by step process, you will discover how various Taikun
resources are declared and managed using Terraform.\\

All your work will be done in the \texttt{workshop/} directory. These are its initial contents.
\begin{raw}
./workshop/
|-- main.tf
|-- taikun_auth.auto.tfvars
|-- variables.tf
\end{raw}
\texttt{main.tf} contains the Provider configuration,
namely its source address and what credentials to use.
You will not need to edit this file.
\begin{tf}
# main.tf
terraform {
  required_providers {
    taikun = {
      source = "itera-io/taikun"
      version = "1.0.0"
    }
  }
}

provider "taikun" {
  email    = var.taikun_email
  password = var.taikun_password
}
\end{tf}
Terraform reads its configuration from all the files with the extension \texttt{.tf},
in the working directory.
Having the provider configuration in \texttt{main.tf} is simply a convention.\\

During this workshop, each task should be coded in a separate config file.
At the end of the workshop, your directory will be organized as such:
\begin{raw}
./workshop/
|-- main.tf
|-- taikun_auth.auto.tfvars
|-- task0.tf
|-- task1.tf
|-- task2.tf
|-- task3.auto.tfvars
|-- task3.tf
|-- task4.tf
|-- task5.tf
|-- task6.tf
|-- task7.tf
|-- variables.tf
|-- users.auto.tfvars
|-- users.tf
\end{raw}

\begin{warn}
  Throughout this workshop, in order to avoid conflicts and
  to permit easy clean up of leftover infrastructure, please follow the naming conventions used below.\\

  All resources names will follow the format \texttt{tfws-<firstname>-<s>}, where:
  \begin{itemize}
    \item \texttt{<firstname>} is your firstname
    \item \texttt{<s>} is an arbitrary string of your choosing
  \end{itemize}
  For example, Jane Doe might name their Taikun project \texttt{tfws-jane-mysuperproject}.
\end{warn}

\subsection{Authentication}\label{sec:auth}
In order to complete the tasks that follow, you will need to provide Taikun credentials to Terraform.
You will be using a Partner account as some of the tasks, such as creating an organization,
require Partner privileges.\\

Input variables will be explained later in the workshop.
For now, simply edit \texttt{taikun\_auth.auto.tfvars}
and replace the values of \texttt{taikun\_email} and \texttt{taikun\_password}
with the credentials provided on the \textbf{\#terraform-workshop} Slack channel.
\begin{tf}
# taikun_auth.auto.tfvars
taikun_email = "jane.doe@itera.io"
taikun_password = "PassWord123"
\end{tf}
To find out more about providing sensitive data to Terraform, see this \href{https://learn.hashicorp.com/tutorials/terraform/sensitive-variables}{Hashicorp tutorial}.
