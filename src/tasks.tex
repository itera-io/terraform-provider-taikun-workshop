\section{Documentation}
Once the provider has been added to the Terraform Registry, the documentation will be available online there.
For the purpose of this workshop, the documentation is being temporarily hosted on \href{https://intuinewin.github.io/taikun-docs/}{Github Pages}.

\section{Tasks}

The \texttt{workshop/} directory has the following content.
\begin{raw}
./workshop/
|-- main.tf
|-- taikun_auth.auto.tfvars
|-- variables.tf
\end{raw}
\texttt{main.tf} contains the Provider configuration, namely where it is installed
and what credentials to use. You will not need to edit this file.
\begin{tf}
terraform {
  required_providers {
    taikun = {
      source = "itera-io/dev/taikun"
    }
  }
}

provider "taikun" {
  email    = var.taikun_email
  password = var.taikun_password
}
\end{tf}
In fact, Terraform reads its configuration from all the files with the extension \texttt{.tf},
in the working directory. The extension \texttt{.tfvars} will be discussed later.
Having the provider configuration in \texttt{main.tf} is simply a convention.
During this workshop, each task should be coded in a separate config file,
such that, at the end of the workshop, your directory is organized as such:
\begin{raw}
./workshop/
|-- main.tf
|-- taikun_auth.auto.tfvars
|-- task0.tf
|-- variables.tf
\end{raw}
% TODO update as tasks are added


\begin{warn}
  Throughout this workshop, in order to avoid conflicts and
  to permit easy clean up of leftover infrastructure, please follow the naming conventions used below.\\

  All resources names will follow the format \texttt{tfws-<firstname>-<s>}, where:
  \begin{itemize}
    \item \texttt{<firstname>} is your firstname
    \item \texttt{<s>} is an arbitrary string of your choosing
  \end{itemize}
  For example, Jane Doe might name their Taikun project \texttt{tfws-jane-mysuperproject}.
\end{warn}

\subsection{Authentication}
In order to complete the tasks that follow, you will need to provide Taikun credentials to Terraform.
You will be using a Partner account, as some of the tasks, such as creating an organization,
require Partner privileges.\\

We will explain how input variables work later in the workshop.
For now, simply edit \texttt{taikun\_auth.auto.tfvars}
and replace the values of \texttt{taikun\_email} and \texttt{taikun\_password}
with the credentials provided on the \textbf{\#terraform-workshop} Slack channel.
\begin{tf}
taikun_email = "jane.doe@itera.io"
taikun_password = "PassWord123"
\end{tf}
To find out more about providing sensitive data to Terraform, see this \href{https://learn.hashicorp.com/tutorials/terraform/sensitive-variables}{Hashicorp tutorial}.

\subsection{Task 0}\label{sec:task0}

\begin{note}
For this task, please write your code in the file \texttt{task0.tf}
at the root of the \texttt{workshop/} directory.
\end{note}

This objective of this first task is to create an Organization.
All resources created in the future will be part of this organization.
As this is the first task, you will be guided through every step of the process.\\

\subsubsection{Organization resource}
Before you do anything, start by preparing your working directory for other commands.
\begin{shell}
terraform init
\end{shell}
\begin{info}
\texttt{terraform init} only needs to be run once when starting a new project or after updating
the list of providers to use.
\end{info}
If all went well, you should see the following message.
\begin{raw}
Initializing the backend...
[...]
Terraform has been successfully initialized!
\end{raw}
If you got an error of the following type, the Taikun provider was not correctly installed.
\begin{raw}
| Error: Failed to query available provider packages
|
| Could not retrieve the list of available versions for provider itera-io/dev/taikun
| itera-io: Failed to request discovery document: ...
| lookup itera-io: no such host
\end{raw}

Once Terraform has been initialized correctly, you can declare your Organization resource.
Create \texttt{task0.tf} and write the following configuration block to it.
\begin{tf}
resource "taikun_organization" "myorg" {
  name          = "tfws-<firstname>-<s>"
  full_name     = "<s>"
  discount_rate = 120
}
\end{tf}
Be sure to replace \texttt{<firstname>} with your firstname and \texttt{<s>} with
names of your choosing. You can also choose another label instead of \texttt{myorg}.

\begin{info}
Notice the syntax of the configuration block, as we are creating a resource,
we begin with the keyword \texttt{resource}, followed by its type between double quotes.
The type of resource is always lowercase and prefixed by the name of the provider,
thus \texttt{"taikun\_organization"}.
Finally we have a label, it must be unique for this type of resource, and is used
to refer to this specific resource, as you will find out later.
Watch out, this label does not correspond to the name of the resource in Taikun.\\

We then define three arguments: \texttt{name}, \texttt{full\_name} and \texttt{discount\_rate}.
On the left side of the equals sign is the argument's identifier, on the right is its value.
See the \href{https://intuinewin.github.io/taikun-docs/resources/organization.html}{documentation of Taikun's organization resource} for a full list of arguments, i.e. the resource's \textit{schema}.
Here we did not bother to define any of the optional arguments, only the required ones.\\

Labels and argument names can contain letters, digits, underscores and hyphens and may not start with a digit.
\end{info}
Now apply your changes.
\begin{shell}
terraform apply
\end{shell}
You should get a validation error.
\begin{raw}
Error: expected discount_rate to be in the range (0.000000 - 100.000000), got 120.000000
\end{raw}
Now fix the discount rate so it is in the range 0-100 and run \texttt{apply} once more.
Terraform will display a list of resources to create.
After checking the plan is correct, type \texttt{yes} to confirm it.
\begin{info}
You should notice a file \texttt{terraform.tfstate} in your working directory,
Terraform uses this file to keep track of the state, do not modify or delete it.
\end{info}
You may now list the resources in Terraform's state.
\begin{shell}
terraform state list
\end{shell}
\begin{raw}
taikun_organization.myorg
\end{raw}
You can also display their content.
\begin{shell}
terraform show
\end{shell}
\begin{raw}
# taikun_organization.myorg:
resource "taikun_organization" "myorg" {
    created_at                       = "2021-11-05T14:00:50Z"
    discount_rate                    = 42
    full_name                        = "Jane Doe's organization"
    id                               = "6383"
    is_read_only                     = false
    lock                             = false
    managers_can_change_subscription = true
    name                             = "tfws-jane-org"
    partner_id                       = "119"
    partner_name                     = "TF-CI"
    projects                         = 0
    servers                          = 0
}
\end{raw}
\begin{info}
Try running \texttt{apply} again, Terraform will refresh its state, and, as long as
nothing has changed, tell you that no changes are needed.
\end{info}

\subsubsection{Organization datasource}
This section is not required to complete the workshop and serves only
to introduce Terraform's datasources. If you are already familiar with
them, you can skip to \fullref{sec:task1}.

\subsection{Task 1}\label{sec:task1}
\blindtext{}
