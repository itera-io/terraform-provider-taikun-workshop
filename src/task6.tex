\subsection{Task 6: Project}\label{sec:task6}

\begin{note}
For this task, please write all your code in the file \texttt{task6.tf}
at the root of the \texttt{workshop/} directory.
\end{note}

Finally, you can declare a project resource.
However, as flavors must be bound to the project,
you must first fetch a list of suitable flavors.\\

To do this, declare a \href{https://intuinewin.github.io/taikun-docs/data-sources/flavors.html}{flavors datasource}.
Datasources, as opposed to resources, only fetch information from providers and do not create any resources.
Add the following block to \texttt{task6.tf}.
\begin{tf}
data "taikun_flavors" "small" {
  # FIXME
}
\end{tf}

\begin{tip}
As you are declaring a datasource and not a resource, the block begins with the keyword \texttt{data} instead of \texttt{resource}.
Once again, the type of datasource is in lowercase and must be prefixed by the name of the provider.
Finally, the label \texttt{"small"} is used to designate this datasource.
\end{tip}

Edit the datasource to search for flavors with 4 or fewer CPUs and no more than 8GB of RAM.
Set its cloud credential ID to that of the cloud credential created in \fullref{sec:task3}.\\

Then declare a local value \texttt{flavors} to be the list of names of the flavors read
by the datasource.
See the \href{https://www.terraform.io/docs/language/values/locals.html}{Terraform documentation}
to know more about local values.
\begin{tf}
locals {
  flavors = [for flavor in data.taikun_flavors.small.flavors : flavor.name]
}
\end{tf}
This will allow you to refer to the list of flavor names with \texttt{local.flavors},
which will be useful when defining the project.\\

You now have everything you need to create a project in Taikun.
\href{https://intuinewin.github.io/taikun-docs/resources/project.html}{Here} is its documentation.\\

These are the requirements for the project resource:
\begin{itemize}
  \item As all previous resources, it must belong to the organization created in \fullref{sec:task0}.
  \item It must use the kubernetes profile defined in \fullref{sec:task1}.
  \item Its alerting profile must be the one defined \fullref{sec:task2}.
  \item It should use the cloud credentials defined in \fullref{sec:task3}.
  \item Monitoring must be enabled.
  \item It should have one bastion, one kubemaster and one kubeworker.
\end{itemize}

\begin{tip}
  To access the first element of a list, use the syntax \texttt{list[index]}.
  For example, to get the first flavor read by the \texttt{flavors} datasource,
  use \texttt{local.flavors[0]}.
\end{tip}
\pagebreak

After creating a project with Terraform, one may want to know its access IP.
Once you have declared the project resource, add the following output block to \texttt{task6.tf}.
\begin{tf}
output "project_access_ip" {
  value = resource.taikun_project.<project-label>.access_ip
}
\end{tf}
Make sure to replace \texttt{<project-label>} with the label you gave your project.
This will display the project's access IP once it has been created.-
\begin{tip}
Output values are a way for the user to output a specific value from Terraform's state.\\

An output value block begins with the keyword \texttt{output} followed by a unique label.
By setting the \texttt{value} argument, the user can decide which value to output.\\

To know more about output values, see 
\href{https://www.terraform.io/docs/language/values/outputs.html}{Terraform's documentation}.
\end{tip}

You can now apply your changes.
\begin{hint}
  The Taikun provider was built with the knowledge that you are probably getting hungry at this point.
  Thus, you can expect to wait about 30 minutes for your project to be in \texttt{Ready} state.
  Please take this time to enjoy the refreshments available in the office, or whatever is
  in your fridge if you are working from home.
\end{hint}
