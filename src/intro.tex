\section{Introduction}
The purpose of this workshop is to introduce you to
\href{https://www.terraform.io/intro/index.html}{Terraform}
and the \href{https://registry.terraform.io/providers/itera-io/taikun/latest}{Terraform Provider for Taikun}.
The latter will allow you to use Terraform to manage resources in Taikun.

\section{How to read this document}

\begin{itemize}
  \item Text in this form is to be typed, as is, on the command line.
\begin{shell}
cd workshop/
ls
echo Hello!
\end{shell}
\item This form of text shows screen output, usually the output of commands.
\begin{raw}
task_00/
task_01/
...
Hello!
\end{raw}
\item This format is for code in Terraform's configuration language,
\href{https://www.terraform.io/docs/language/syntax/configuration.html}{HashiCorp Configuration Language (HCL)}.
\begin{tf}
resource "aws_instance" "example" {
  ami = "abc123"

  network_interface {
    # ...
  }
}
\end{tf}
\item You may wish to skip reading information blocks if you are already familiar with
Terraform and its configuration syntax.
\begin{tip}
  Some information about Terraform...
\end{tip}
\end{itemize}
