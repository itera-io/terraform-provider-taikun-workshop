\begin{important}{Important}
  This document is for Windows users, if you are a Linux or MacOS user, please read \texttt{terraform-taikun-workshop-unix.pdf} instead.
\end{important}

\section{Introduction}
% TODO introduction to Terraform

The Terraform Provider for Taikun is hosted on \href{https://github.com/itera-io/terraform-provider-taikun}{GitHub}.
\blindtext{}

\section{How to read this document}

Text in the following style of boxes are commands to type on the command line.
\begin{shell}
cd workshop/
ls
echo Hello!
\end{shell}
While text in the following style of boxes correspond to the expected output of commands.
\begin{raw}
task_00/
task_01/
...
Hello!
\end{raw}

\section{Requirements}
You must have Terraform version 0.14 or newer installed (see \href{https://www.terraform.io/downloads.html}{Download Terraform}).

\section{Installing the provider locally}
Until the provider is listed on Terraform's plugin \href{https://registry.terraform.io/browse/providers}{registry}, the provider must be installed locally.
Start by cloning the GitHub repo.
\begin{shell}
git clone git@github.com:itera-io/terraform-provider-taikun.git
\end{shell}
Then run the following commands to install the provider on your machine.
\begin{shell}
cd terraform-provider-taikun
make dockerinstall
\end{shell}
The provider will be installed at the following location.
\begin{raw}
~/.terraform.d/plugins/itera-io/dev/taikun/0.1.0/linux_amd64/terraform-provider-taikun
\end{raw}

\section{Setup}
Terraform config files with the minimal required configuration are provided for each task.
Download the \texttt{workshop.zip} file from [insert method of distribution] % slack channel? git repo?
and extract its contents in the directory of your choice.\\

To do so from the command line, type the following command.
\begin{shell}
unzip workshop.zip
\end{shell}

Then switch into the extracted \texttt{workshop} directory.
% TODO complete this as tasks are added
\begin{shell}
cd workshop/
\end{shell}
Its contents should be as follows.
\begin{raw}
|-- task_00/
    |-- main.tf
    |-- taikun_credentials.tfvars
    |-- variables.tf
|-- setup.sh*
\end{raw}
Each task has a dedicated directory, when moving on to a new task, please switch into the corresponding directory.

\section{Authentication}
In order to complete the tasks that follow, you will need to provide Taikun credentials to Terraform.
As some of the tasks require Partner privileges, please use a Partner or Admin account.\\

We will not explain how to provide sensitive input variables to Terraform as those concerns are outside the scope of this
workshop. % or should we let people do this setup themselves?
To find out more about this topic, see this \href{https://learn.hashicorp.com/tutorials/terraform/sensitive-variables}{Hashicorp tutorial}.
A script is thus provided to manage Taikun authentication.
Run it and provide your credentials as follows. The script only needs to be run once.
\begin{shell}
./setup.sh
\end{shell}
\begin{raw}
Enter your Taikun email:
> jane.doe@itera.io
Enter your Taikun password:
> password123
\end{raw}

\section{Tasks}
\subsection{Task 0}

\begin{tip}{Note}
  For this task, please switch into the directory \texttt{workshop/task\_00}.
\end{tip}

\blindtext{}
